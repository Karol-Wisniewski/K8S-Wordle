\documentclass[12pt,a4paper]{article}
\usepackage[utf8]{inputenc}
\usepackage{dsfont} 
\usepackage[polish]{babel}
\usepackage{amsmath}
\usepackage{graphicx}
\usepackage[top=1in, bottom=1.5in, left=1.25in, right=1.25in]{geometry}

\usepackage{subfig}
\usepackage{multirow}
\usepackage{multicol}
\graphicspath{{Imagens/}}
\usepackage{xcolor,colortbl}
\usepackage{float}

\newcommand \comment[1]{\textbf{\textcolor{red}{#1}}}

%\usepackage{float}
\usepackage{fancyhdr} % Required for custom headers
\usepackage{lastpage} % Required to determine the last page for the footer
\usepackage{extramarks} % Required for headers and footers
\usepackage{indentfirst}
\usepackage{placeins}
\usepackage{scalefnt}
\usepackage{xcolor,listings}
\usepackage{textcomp}
\usepackage{color}
\usepackage{verbatim}
\usepackage{framed}

\definecolor{codegreen}{rgb}{0,0.6,0}
\definecolor{codegray}{rgb}{0.5,0.5,0.5}
\definecolor{codepurple}{HTML}{C42043}
\definecolor{backcolour}{HTML}{F2F2F2}
\definecolor{bookColor}{cmyk}{0,0,0,0.90}  
\color{bookColor}

\lstset{upquote=true}

\lstdefinestyle{mystyle}{
	backgroundcolor=\color{backcolour},   
	commentstyle=\color{codegreen},
	keywordstyle=\color{codepurple},
	numberstyle=\numberstyle,
	stringstyle=\color{codepurple},
	basicstyle=\footnotesize\ttfamily,
	breakatwhitespace=false,
	breaklines=true,
	captionpos=b,
	keepspaces=true,
	numbers=left,
	numbersep=10pt,
	showspaces=false,
	showstringspaces=false,
	showtabs=false,
}
\lstset{style=mystyle}

\newcommand\numberstyle[1]{%
	\footnotesize
	\color{codegray}%
	\ttfamily
	\ifnum#1<10 0\fi#1 |%
}

\definecolor{shadecolor}{HTML}{F2F2F2}

\newenvironment{sqltable}%
{\snugshade\verbatim}%
{\endverbatim\endsnugshade}

% Margins
\addtolength{\footskip}{0cm}
\addtolength{\textwidth}{1.4cm}
\addtolength{\oddsidemargin}{-.7cm}

\addtolength{\textheight}{1.6cm}
%\addtolength{\topmargin}{-2cm}

% paragrafo
\addtolength{\parskip}{.2cm}

% Set up the header and footer
\pagestyle{fancy}
\rhead{\hmwkAuthorName} % Top left header
\lhead{\hmwkClass: \hmwkTitle} % Top center header
\rhead{\firstxmark} % Top right header
\lfoot{Karol Wiśniewski} % Bottom left footer
\cfoot{} % Bottom center footer
\rfoot{} % Bottom right footer
\renewcommand{\headrulewidth}{1pt}
\renewcommand{\footrulewidth}{1pt}

    
\newcommand{\hmwkTitle}{Wordle} % Tytuł projektu
\newcommand{\hmwkDueDate}{\today} % Data 
\newcommand{\hmwkClass}{Technologie chmurowe} % Nazwa przedmiotu
\newcommand{\hmwkAuthorName}{Karol Wiśniewski} % Imię i nazwisko

% trabalho 
\begin{document}
% capa
\begin{titlepage}
    \vfill
	\begin{center}
	\hspace*{-1cm}
	\vspace*{0.5cm}
    \includegraphics[scale=0.55]{imagens/loga.png}\\
	\textbf{Uniwersytet Gdański \\ [0.05cm]Wydział Matematyki, Fizyki i Informatyki \\ [0.05cm] Instytut Informatyki}

	\vspace{0.6cm}
	\vspace{4cm}
	{\huge \textbf{\hmwkTitle}}\vspace{8mm}
	
	{\large \textbf{\hmwkAuthorName}}\\[3cm]
	
		\hspace{.45\textwidth} %posiciona a minipage
	   \begin{minipage}{.5\textwidth}
	   Projekt z przedmiotu technologie chmurowe na kierunku informatyka profil praktyczny na Uniwersytecie Gdańskim.\\[0.1cm]
	  \end{minipage}
	  \vfill
	%\vspace{2cm}
	
	\textbf{Gdańsk}
	
	\textbf{\hmwkDueDate}
	\end{center}
	
\end{titlepage}

\newpage
\setcounter{secnumdepth}{5}
\tableofcontents
\newpage

\section{Opis projektu}
\label{sec:Project}

Pewna firma o nazwie "Language Learners Inc." zajmująca się edukacją językową postanowiła wdrożyć innowacyjne narzędzie, które pomogłoby osobom uczącym się języków obcych rozwijać swoje umiejętności słownictwa w sposób interaktywny i angażujący. Przez wiele lat nauki słownictwa w tradycyjny sposób było monotonne i nieprzyjemne, co często powodowało brak motywacji u uczniów.

Aby temu zaradzić, firma Language Learners Inc. zdecydowała się zamówić aplikację o nazwie Wordle, która pozwoliłaby użytkownikom na naukę słownictwa w ciekawy i zabawny sposób. Celem aplikacji Wordle jest dostarczenie interaktywnego doświadczenia, w którym użytkownicy mogliby zgadywać słowa na podstawie podpowiedzi i zyskiwać punkty za poprawne odpowiedzi.

\subsection{Opis architektury - 8 pkt}
\label{sec:introduction}
Aplikacja Wordle jest oparta na Kubernetes, który jest otwartoźródłowym systemem zarządzania kontenerami. Kubernetes umożliwia skalowalność, zarządzanie, automatyzację i odizolowanie aplikacji w kontenerach. Dodatkowo, aby uprościć poruszanie się po aplikacji, adres IP klastra jest utożsamiany z adresami project.baw (main), api.project.baw (backend), keycloak.project.baw (keycloak). Każdy z tych adresów jest odpowiednio interpretowany dzięki dodatkowej warstwie Ingress, która działa bardzo podobnie do default.conf serwera nginx. Jest to "interpreter" poruszania się po aplikacji. Sztuczne domeny stworzone zostały za pomocą konfiguracji pliku "hosts" w katalogach źródłowych systemu Windows. Działają one na takiej samej zasadzie jak 127.0.0.1 i "localhost".

\subsection{Opis infrastruktury - 6 pkt}
\label{sec:Users}

%Powinien zawierać informacje na temat środowiska, w którym aplikacja będzie działać, a także wykorzystanych narzędzi i platformy chmurowe. Ważne jest również zwrócenie uwagi na wykorzystanie zasobów, takich jak sieci i pamięci masowej.
Cały klaster Kubernetes jest hostowany przez maszynę wirtualną, stworzoną za pomocą narzędzia minikube. Jej specyfikacja to: 10GB pamięci masowej, 6GB pamięci operacyjnej (RAM) oraz 2 wirtualne CPU.\\

\subsection{Opis komponentów aplikacji - 8 pkt}
\label{sec:FunctionalConditions}

%Powinna zawierać informacje na temat komponentów aplikacji, takich jak serwisy, aplikacje i bazy danych. W szczególności należy zwrócić uwagę na sposoby ich wdrażania, konfiguracji i zarządzania.

\textbf{Frontend} (React): Moduł odpowiedzialny za interfejs użytkownika aplikacji. Zbudowany w oparciu o framework React. Został wdrożony poprzez customowy Dockerfile, zbudowany na Dockerze samej maszyny wirtualnej, na której uruchomiony jest klaster. Jego ścieżki obsługuje dodatkowy serwer nginx z dostosowanym plikiem konfiguracyjnym default.conf. \\
\\
\textbf{Backend} (Express): Moduł odpowiedzialny za logikę biznesową aplikacji. Wykorzystuje framework Express do obsługi żądań HTTP. Tak samo jak frontend, został on wdrożony poprzez customowy Dockerfile na maszynie wirtualnej. Jego endpointy są chronione przez bibliotekę keycloak-connect i wymagają zalogowania się w serwisie. Dodatkowo, jeden z nich wymaga od użytkownika statusu admina.\\
\\
\textbf{Keycloak}: Moduł odpowiedzialny za autentykację i zarządzanie tożsamościami użytkowników. Keycloak jest otwartoźródłowym narzędziem do zarządzania tożsamościami opartym na protokole OpenID Connect.\\
\\
\textbf{PostgreSQL}: Moduł bazy danych, który przechowuje dane związane z tożsamościami użytkowników Keycloak. Dzieli swoje dane z PVC (Persistent Volume Claim) na klastrze, aby konfiguracja serwisu keycloak pozostawała taka sama, nawet po całkowitym restarcie klastra Kubernetes.\\

%W sumie aplikacja składa się z 4 Podów, 3 ConfigMap, 1 Ingressa, 4 Serwisów, i 1 PVC (domyślne zasoby Kubernetesa nie zostały wliczone).

\subsection{Konfiguracja i zarządzanie - 4 pkt}
\label{sec:NonFunctionalConditions}

%Powinna zawierać informacje na temat konfiguracji i zarządzania aplikacją na poziomie klastra Kubernetes.
\textbf{Frontend}: korzysta wyłącznie z jednego serwisu, który definiuje na jakim porcie ma stać ta część klastra.\\
\\
\textbf{Backend}: wykorzystuje jedną ConfigMapę oraz jeden serwis. ConfigMapa ma za zadanie dostarczyć backendowi niezbędne zmienne, takie jak FRONT-ORIGIN (skąd będą przychodzić zapytania do API) oraz kilka zmiennych konfiguracyjnych dla biblioteki keycloak-connect. Serwis zaś, ma za zadanie zdefiniować port, na którym będzie nasłuchiwał backend.\\
\\
\textbf{Postgres}: w przypadku tej warstwy potrzebna była porządna konfiguracja. Wykorzystuje ona jeden Secret, jedną ConfigMapę oraz jeden PVC (Persistent Volume Claim). Secret to bezpieczne miejsce dla zmiennych środowiskowych wykorzystywanych przez dany serwis. Przechowuje takie dane jak nazwa użytkownika bazy danych, hasło oraz nazwę samej bazy. ConfigMapa w tym przypadku jest inicjalizowana przez customowy skrypt bashowy, który bazuje na query SQL'owym, tworzącym bazę Keycloak, potrzebną dla samego Keycloaka do persystencji danych. Tak jak w poprzednich przypadkach, Serwis ma za zadanie zdefiniować port nasłuchiwania dla danej warstwy.\\
\\
\textbf{Keycloak}: ta warstwa korzysta z jednego Serwisu i jedego Secreta. Serwis dzieli takie samo zadanie, jak w przypadku reszty warstw. Secret przechowuje delikatne dane do konfiguracji samego serwisu Keycloak oraz jego połączenia z wcześniej opisaną bazą PostgreSQL.\\

\subsection{Zarządzanie błędami - 2 pkt}
\label{sec:ERD} 

%Powinna zawierać informacje na temat sposobów zarządzania błędami aplikacji oraz sposobów monitorowania i reagowania na awarie.


\subsection{Skalowalność - 4 pkt}
\label{sec:ExamplesSection}

%Skalowalność jest kluczowa w architekturze aplikacji opartej na Kubernetes. Należy opisać, jak aplikacja może być skalowana, w jaki sposób skalowanie jest monitorowane i jakie narzędzia są wykorzystywane w tym celu.

\subsection{Wymagania dotyczące zasobów - 2 pkt}
\label{sec:ExampleTables}

%Powinna zawierać informacje na temat wymagań dotyczących zasobów dla każdego komponentu aplikacji, takie jak ilość pamięci RAM, CPU, miejsce na dysku, itp. Należy również opisać, jakie są oczekiwania dotyczące wydajności i czasu odpowiedzi dla aplikacji.

\textbf{Frontend}: wykorzystuje 200MB pamięci podręcznej i 0.2 CPU.\\
\\
\textbf{Backend}: podobnie jak frontend, potrzebuje 0.2CPU, ale za to już 600MB pamięci podręcznej.\\
\\
\textbf{Keycloak}: wykorzystuje najwięcej zasobów, bo aż 0.4 CPU i prawie 2GB pamięci podręcznej. Są to minimalne wymagania dla tego serwisu.\\
\\
\textbf{Postgres}: ma niewielkie wymagania, wykorzystuje zaledwie 0.1 CPU i 128MB pamięci podręcznej.\\


\subsection{Architektura sieciowa - 4 pkt}
\label{sec:ExampleResults}

%powinna zawierać informacje na temat architektury sieciowej aplikacji, w tym sposobu konfiguracji sieci w klastrze Kubernetes, wykorzystywanych protokołów i narzędzi do zarządzania siecią.

\noindent
%\textbf{Każdy odnośnik, który znajdzie się w literaturze musi mieć swoje odwołanie w projekcie. - 2 pkt} 

\bibliographystyle{amsplain}
\bibliography{references.bib}
\nocite{*}

\end{document}
